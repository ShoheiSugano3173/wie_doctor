%==============================================================
%=================  �_���v�|�@�i�@�p��@�j ========================
%==============================================================
\noindent In this research, I developed an excretion detection device using sensor technology.
As a result, this research succeeded in developing a non-wearing and noninvasive sheet type excretion detector using odor sensors.
  As of 2018 Japan faces the problem of population decline and rapid aging.
Although the population receiving nursing-care insurance services also increases, the turnover rate of carers in the nursing care industry is high and manpower shortage is also pronounced.
In nursing care sites where personnel shortage is serious, not only the conventional nursing care system but also the demand for the introduction of nursing care robots technology is increasing.
However, at the actual site, the introduction of nursing care robots has not gone through full-scale.
One of the reasons is that products developed while deviating from the needs of the nursing care site exist, and there are needs, but functions that are hard to understand and difficult to handle by users are provided.
Against this backdrop, I focused on excretion service even among nursing care work.
In this paper, we aimed at developing advanced welfare tools that can easily be used by anyone, by contributing to the penetration of advanced technology in the nursing care industry by utilizing sensor technology and IoT technology.
Many researchers have been studying excretion detection using sensor technology. In addition, many manufacturers develop and release excretion detection products, but They have not been able to develop widely pervasive products.
If excretion detector that can easily use excretion assistance is developed, I view that it is possible for inexperienced persons and foreign workers to take care at a level equivalent to that of someone with experience of care.
Moreover, by realizing efficient nursing care, I expected to reduce the burden on care workers and further improve service in surplus time.
  In this thesis, first, I verified nursing care services that require the introduction of robot technology. Therefore, a questionnaire survey was conducted for nursing care workers engaged in nursing care sites.
As a result, it was shown that excretion work is a burden for nursing care workers at many nursing care sites. 
In addition, it was found that the excretion service spends about 20\% of the whole time.
The reason that many time is spent on excretion work was found to be individual differences in excretion timing and frequency.
Then, in confirmation in the scheduled replacement service, it was found that even if the nursing care professional confirms the diaper is a situation where there is no excretion in one or it is leaking from one.
Similarly, in the questionnaire survey of the sense of burden on excretion work, the proportion of caregivers who felt as a burden was about 70\% of the total.
Based on the questionnaire results, we investigated the points required for excretion work while referring to existing products.
As a result, attention was paid to three points of "non-wearing", "the detection of feces is possible" and "considering the cost aspect", I used it as a concept of excretion detection sensor.
  Next, in order to develop products that can withstand practical use at nursing care sites, I constructed and verified an algorithm to detect excretion using sensor values.
Validation experiments were conducted for young people and elderly people.
As a result, I found that the algorithm to detect youth excretion cannot be applied to the elderly.
For this reason, I reconstructed an algorithm that matched the excretion characteristics of elderly people based on the experimental results.
  In this research, a web application was also developed to accumulate excretion data and create individual excretion patterns.
It was found that this system could prevent a situation where there is no excretion in a daiper or it is leaking from one to provide higher quality nursing care work.
In addition, I devised a method to solve the mismatch between consciousness of both developers and users, and to develop and disseminate products that can be utilized at the site, together with development methods.