%==============================================================
%=================  �_���v�|�@�i�@�p��@�j ========================
%==============================================================
\noindent In this research, I developed an excretion detection device using sensor technology.  
As a result, this research succeeded in developing a non-wearing and noninvasive sheet type excretion detector using odor sensors.  
As of 2018 Japan faces the problem of population decline and rapid aging.  
As the population receiving nursing-care insurance services increase, the turnover rate of carers in the nursing care industry remains high and therefore, manpower shortage is also pronounced.  
In nursing care sites where personnel shortage is serious, not only is the conventional nursing care system needed, but the introduction of nursing care robot technology is also in high demand .  
However, at actual site, the introduction of nursing care robots has not gone through full-scale.  
One of the reasons is that products are developed without understanding the actual needs of the nursing care sites.   Therefore, functions that are hard to understand and difficult to handle by users are unfortunately being provided.  
In this dissertation, I aimed to develop advanced welfare tools that can easily be used by everyone, and therefore aim to contribute to the penetration of advanced technology in the nursing care industry.   I focused on sensor technology and IoT technology to that applies to excretion services amongst nursing care work.  
Many researchers have been studying excretion detection using sensor technology and with its results, many manufacturers have developed and released excretion detection products in the past years;.   yet they have not been able to develop widely pervasive products.  
If excretion detectors that can easily be used for excrete assistance, I view that it is possible for inexperienced people and foreign workers to take care at a level equivalent to that of someone with experience of care.  
Moreover, by realizing efficient nursing care, I expect to reduce the burden on care workers that enables further improvement of care services.  
In this dissertation, first, I verified nursing care services that require the introduction of robot technology.   Therefore, a questionnaire survey was conducted for nursing care workers engaged in nursing care sites.  
As a result, it was shown that excretion care work is being a non-negligible burden for nursing care workers at many nursing care sites.   In addition, it was found that the excretion service spends about 20% of the whole work hour.  
The reason that much time is spent on excretion work is because of the individual differences in excretion timing and frequency.  
In confirmation of the scheduled replacement service, I found that even if the nursing care professional checks one�fs diaper, there were situations where there were no excretion or on the other hand, was leaking from the diaper.  
Similarly, in the questionnaire survey, the proportion of caregivers who felt that the excretion work being a burden was about 70\%.   Based on the questionnaire results, we investigated the considerations required for excretion work while referring to existing products.  
As a result, attention was paid to three key points.  "Non-wearing", "Detection of feces" and "Consideration of cost".   With this three aspects, I conceptualized an excretion detection sensor.   As a result, I adopted a gas sensor to detect excretion through the associated odor.   
Next, in order to develop products that can withstand practical use at nursing care sites, I constructed and verified an algorithm to detect excretion using sensor values.   Validation experiments were conducted for young people and elderly people and as a result, I found that the algorithm to detect youth excretion cannot be applied to the elderly.  
For this reason, I reconstructed an algorithm that matched the excretion characteristics of elderly people based on experimental results.  
In addition, a web application was also developed to accumulate excretion data and create individual excretion patterns.  
It was found that this system could prevent a situation where there is no excretion at the scheduled diaper change timing or leaking of feces to provide higher quality nursing care work.  
In addition, I devised a method to solve the mismatch between consciousness of both developers and users, and to develop and disseminate products that can be utilized at the site, together with development methods.    Detailed discussions on these points are also provided in this dissertation.  
